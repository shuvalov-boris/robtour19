Программное обеспечение разработано в Arduino I\+DE и состоит из следующих модулей (классов)\+: \mbox{\hyperlink{robtour19_8ino}{robtour19.\+ino}} -\/ точка входа в программу, координирует работу алгоритмов по сбору фишек и размещению их на мишени. Также содержит реализацию алгоритма движения по черной полосе. \mbox{\hyperlink{class_c_color_tracker}{C\+Color\+Tracker}} позволяет определить цвет стратового поля и отслеживать моменты пересечения синих полос. \mbox{\hyperlink{class_c_drive_control}{C\+Drive\+Control}} управляет пространственным перемещением робота, описывает алгоритм перестроения на черную полосу с фишкой. С\+Gear\+Motor управляет роторным мотором, который используется на ведущей задней оси привода и для подъема каретки с фишками наверх. \mbox{\hyperlink{class_c_hamming_code}{C\+Hamming\+Code}} декодирует сообщение от ИК-\/передатчика. \mbox{\hyperlink{class_c_tower_control}{C\+Tower\+Control}} управляет платформой для захвата фишек, подъёмом платформы и её наклоном вместе с башней.

На диаграмме представлен граф зависимостей модулей проекта. 